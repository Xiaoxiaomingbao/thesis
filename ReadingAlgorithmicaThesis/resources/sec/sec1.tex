\subsection{问题引入}\label{subsec:intro}

本文在 Data Structures for Computing Unique Palindromes in Static and Non-Static Strings~\cite{Mieno2024}
(以下称原文)的基础上,继续探讨静态串和非静态串的唯一回文子串问题。本文首先概述原文的主要内容和算法设计思路,
并调研回文串领域的问题与解决方法,再描述原文中算法的不足之处,并针对不足之处提出改进算法。

本文在概述原文时,着重介绍数据结构的设计。对于原文中涉及的大量引理(Lemma)、定理(Theorem)、命题(Proposition)、
推论(Corollary),本文直接引用其中的结论,不再关注其证明过程,因为原文提供了详尽的证明或证明所在的参考文献。
由于原文中直接引用了一些不太常用的数据结构,如后缀树、van Emde Boas树等,本文会适当补充相关数据结构的定义、实现、性质。

\begin{description}
    \item [MUPS] 给定串$T$,$1 \leq i \leq j \leq ||T||$,回文子串$u = T[i..j]$,如果$u$在$T$中是唯一的,
    $u^{\prime} = T[i + 1..j - 1]$在$T$中不唯一,则称$u$是$T$的极小唯一回文子串(minimal unique palindromic substring).
    \item [interval SUPS] 给定串$T$和区间$[p,q]$,回文子串$v = T[i..j]$,$v$在$T$中唯一且包含区间$[p,q]$,
    任何更短的包含区间$[p,q]$的$T$的回文子串都在$T$中不唯一,则称$v$是对区间$[p,q]$的$T$的最短唯一回文子串
    (shortest unique palindromic substring),记作$v = \mathrm{SUPS}_T ([p,q])$.
    \item[point SUPS] 区间SUPS问题在$p = q$的情况下退化为点的SUPS问题,记作$v = \mathrm{SUPS}_T (p)$.
    \item[maximal palindrome] MUPS向外扩展,即每次起点向前移动一位,终点向后移动一位,直至破坏回文性为止。
    在此过程中可以得到“极大回文子串”(maximal palindrome).
\end{description}

需要注意的是,MUPS是极小的,强调的是MUPS都无法在自身基础上再缩小一点,因为再缩小会违背MUPS的唯一性,
MUPS的定义并不关心每个MUPS与其他MUPS的长度比较。SUPS是最短的,SUPS的定义基于多个包含给定区间或点的回文子串的长度比较,
但这并不意味着SUPS仅有1个。原文中的Theorem 1说明SUPS至多只有4个。

由MUPS的定义可知,MUPS具有唯一性,此外MUPS不存在嵌套的情况,可用反证法论证如下:

取MUPS $M_1 = T[s_1..e_1]$,MUPS $M_2 = [s_2..e_2]$,满足$s_1 \leq s_2 \leq e_1 \leq e_2 $,
且$s_1 \leq s_2$和$e_1 \leq e_2$不能同时取等。找到$M_1$的对称轴$l$,若$M_2$关于$l$对称,则$M_1$显然不满足MUPS定义中的极小性;
若$M_2$不关于$l$对称,则取串$M_3$关于轴$l$与$M_2$对称,由于$M_2$是回文串,$M_3$与$M_2$完全一致,显然违背了$M_2$作为MUPS的唯一性。

原文中并未正式给出“极大回文子串”(maximal palindrome)的定义,但确实使用了这个概念。需要注意的是,极大回文子串是基于MUPS扩展而来的,
每一个MUPS都唯一对应一个极大回文子串。极大回文子串具有唯一性和非嵌套性,可用反证法论证如下:

假设存在两个完全一致的极大回文子串$m_1$和$m_2$,$m_1$对应的MUPS为$M_1$,以$m_1$和$M_1$的相对位置关系在$m_2$中找到回文子串$M_2$,
$M_2$一定与$M_1$不重合,于是违背了$M_1$作为SUPS的唯一性,极大回文子串的唯一性得证。

取极大回文子串$m_1 = T[s_1..e_1]$,极大回文子串$m_2 = [s_2..e_2]$,满足$s_1 \leq s_2 \leq e_1 \leq e_2 $,
且$s_1 \leq s_2$和$e_1 \leq e_2$不能同时取等。找到$m_1$的对称轴$l$,若$m_2$关于$l$对称,则$m_2$不满足极大回文子串定义中的极大性;
若$m_2$不关于$l$对称,则取串$m_3$关于轴$l$与$m_2$对称,由于$m_2$是回文串,$m_3$与$m_2$完全一致,
$m_2$不满足上文已证的极大回文子串的唯一性。

\subsection{数据结构的工具}\label{subsec:tools}

最低共同祖先(lowest common ancestor)

后缀树(suffix tree)

van Emde Boas 树

\subsubsection{最长公共扩展(LCE)问题}\label{subsubsec:lce}

最长公共扩展(longest common extension, LCE)问题是对一个给定的串$T$,给定的整数$i,j$,满足$1 \leq i,j \leq n$,
计算出两个后缀$T[i..n]$和$T[j..n]$的最长公共前缀的长度。

\subsubsection{区域最小值请求(RmQ)、前驱(Predecessor)和后继(Successor)}\label{subsubsec:rmq}

区域最小值请求(range minimum query ,RmQ)是对一个给定的整型数组$A$,$A$上的两个下标$i,j$,满足$i \leq j$,
计算出下标$k$使得$A[k]$为$A[i..j]$上的最小值。注意RmQ数据结构并不要求整型数组有序。

对一个不减的整型数组$B$,给定整型数$x$,定义前驱(Predecessor)是小于$x$的最大值;同理,定义后继(Successor)是大于$x$的最小值。

\subsection{SUPS问题的数据结构}\label{subsec:sups}

\subsection{算法实例}\label{subsec:eg}
