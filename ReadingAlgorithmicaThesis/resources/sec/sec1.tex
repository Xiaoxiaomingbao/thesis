
本文在概述原文时,着重介绍数据结构的设计。对于原文中涉及的大量引理(Lemma)、定理(Theorem)、命题(Proposition)、
推论(Corollary),本文直接引用其中的结论,不再关注其证明过程,因为原文提供了详尽的证明或证明所在的参考文献。
由于原文中直接引用了一些不太常用的数据结构,如后缀树、van Emde Boas树等,本文会适当补充相关数据结构的定义、实现、性质。
对于原文中定义不够明确的概念,本文给出了严格的定义,并描述和论证了相关性质。

\subsection{问题引入}\label{subsec:intro}

\begin{description}
    \item [MUPS] 给定串$T$,$1 \leq i \leq j \leq ||T||$,回文子串$u = T[i..j]$,如果$u$在$T$中是唯一的,
    $u^{\prime} = T[i + 1..j - 1]$在$T$中不唯一,则称$u$是$T$的极小唯一回文子串(minimal unique palindromic substring).
    \item [interval SUPS] 给定串$T$和区间$[p,q]$,回文子串$v = T[i..j]$,$v$在$T$中唯一且包含区间$[p,q]$,
    任何更短的包含区间$[p,q]$的$T$的回文子串都在$T$中不唯一,则称$v$是对区间$[p,q]$的$T$的最短唯一回文子串
    (shortest unique palindromic substring),记作$v = \mathrm{SUPS}_T ([p,q])$.
    \item[point SUPS] 区间SUPS问题在$p = q$的情况下退化为点的SUPS问题,记作$v = \mathrm{SUPS}_T (p)$.
    \item[ME] MUPS向外扩展,即每次起点向前移动一位,终点向后移动一位,直至破坏回文性为止。
    在此过程中可以得到“极小唯一回文子串的极大扩展”(maximal extension of MUPS).
    \item[MP] 对任意一个半整数,即1, 1.5, 2, 2.5 .. , n \footnote{原文中串从1开始标号,本文也采用这种规范。},
    都能求得以其为中心的极大回文子串(maximal palindrome).
\end{description}

需要注意的是,MUPS是极小的,强调的是MUPS都无法在自身基础上再缩小一点,因为再缩小会违背MUPS的唯一性,
MUPS的定义并不关心每个MUPS与其他MUPS的长度比较。SUPS是最短的,SUPS的定义基于多个包含给定区间或点的回文子串的长度比较,
但这并不意味着SUPS仅有1个。原文中的Theorem 1说明SUPS至多只有4个。

由MUPS的定义可知,MUPS具有唯一性,此外MUPS不存在嵌套的情况,可用反证法论证如下:

取MUPS $M_1 = T[s_1..e_1]$,MUPS $M_2 = [s_2..e_2]$,满足$s_1 \leq s_2 \leq e_1 \leq e_2 $,
且$s_1 \leq s_2$和$e_1 \leq e_2$不能同时取等。找到$M_1$的对称轴$l$,若$M_2$关于$l$对称,则$M_1$显然不满足MUPS定义中的极小性;
若$M_2$不关于$l$对称,则取串$M_3$关于轴$l$与$M_2$对称,由于$M_2$是回文串,$M_3$与$M_2$完全一致,显然违背了$M_2$作为MUPS的唯一性。

原文中使用的“极大回文子串”(maximal palindrome)一词,其实兼指ME和MP。这里明确地区分了这两个概念,并在下文探讨两者的性质和联系。

ME具有唯一性和非嵌套性,可用反证法论证如下:

假设存在两个完全一致的ME$m_1$和$m_2$,$m_1$对应的MUPS为$M_1$,以$m_1$和$M_1$的相对位置关系在$m_2$中找到回文子串$M_2$,
$M_2$一定与$M_1$不重合,于是违背了$M_1$作为SUPS的唯一性。ME的唯一性得证。

取ME$m_1 = T[s_1..e_1]$,ME$m_2 = [s_2..e_2]$,满足$s_1 \leq s_2 \leq e_1 \leq e_2 $,
且$s_1 \leq s_2$和$e_1 \leq e_2$不能同时取等。找到$m_1$的对称轴$l$,若$m_2$关于$l$对称,则$m_2$不满足ME定义中的极大性;
若$m_2$不关于$l$对称,则取串$m_3$关于轴$l$与$m_2$对称,由于$m_2$是回文串,$m_3$与$m_2$完全一致,
$m_2$不满足上文已证的ME的唯一性。ME的非嵌套性得证。

显然,MP完全包含了ME。通常情况下,MP包含了大量的空串和单字符串,并有大量重复情况。例如,串$T = \mathrm{babbbabbababb}$,
以2为中心的MP $\mathrm{bab}$和以11为中心的MP $\mathrm{bab}$发生重复,因此两者都不包含在ME中。

事实上,唯一且非空的MP等同于ME。证明如下:

对任意非空的MP,定义“缩小串集”为MP缩小\footnote{类似于扩展,缩小指每次将回文串的起点向后移动一位,终点向前移动一位。}过程中产生的所有回文串的总和。
易知,“缩小串集”包含了所有回文子串,也就包含了所有ME。不唯一的MP缩小产生的回文串也是不唯一的,因而不能产生ME。对唯一的MP进行缩小,缩小的尽头是空串,
必然丧失唯一性。因此,在缩小的过程中一定存在唯一性发生改变的临界,即可找到一个MUPS,则该唯一的MP也就成为一个ME。综上,唯一且非空的MP与ME一一对应。

\subsubsection{最长公共扩展(LCE)与后缀树(Suffix tree)}\label{subsubsec:lce}

对于长度为$n$的串$S$,后缀树(Suffix tree)定义为这样一棵树,满足如下五个条件\cite{suffix}:
\begin{itemize}
    \item 有$n$个子节点,分别编号$1$至$n$
    \item 除根节点外,每个非叶节点都有至少两个子节点
    \item 每条边都用$S$的一个非空子串来标记
    \item 起始于同一节点的两条边不能有同一字符开头的标记
    \item 从根节点到叶节点$i$,将所有边的标记依次拼接起来,可得到后缀$S[i..n]$
\end{itemize}

\begin{figure*}[h]
    \centering
    \includegraphics[width=0.8\textwidth]{resources/fig/Suffix_tree_BANANA}
    \caption{$\mathrm{BANANA\$}$的后缀树,虚线为构建过程中的辅助线\cite{suffix}}
\end{figure*}

最长公共扩展(longest common extension, LCE)问题是对一个给定的串$T$,给定的整数$i,j$,满足$1 \leq i,j \leq n$,
计算出两个后缀$T[i..n]$和$T[j..n]$的最长公共前缀的长度。

\subsubsection{区域最小值请求(RmQ)及变体LogRmQ}\label{subsubsec:rmq}

区域最小值请求(range minimum query ,RmQ)是对一个给定的整型数组$A$,$A$上的两个下标$i,j$,满足$i \leq j$,
计算出下标$k$使得$A[k]$为$A[i..j]$上的最小值。注意RmQ数据结构并不要求整型数组有序。

\subsection{前驱(Predecessor)、后继(Successor)与van Emde Boas树}\label{subsec:van}

对一个不减的整型数组$B$,给定整型数$x$,定义前驱(Predecessor)是小于$x$的最大值;同理,定义后继(Successor)是大于$x$的最小值。

\subsection{Inoue算法概述}\label{subsec:inoue}

\subsection{原文对Inoue算法的改进}\label{subsec:progress}

\subsection{原文算法的示例}\label{subsec:example}
