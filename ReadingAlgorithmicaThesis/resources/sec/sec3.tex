
原文基于Inoue算法提出了区间SUPS问题和点的SUPS问题的优化算法,只需要$O(n)$空间的数据结构,$O(n)$的预处理时间,即可在常数时间内计算出SUPS问题的解。
如果只使用大O记法,不考虑时空复杂度的系数,那么这两类问题的算法已经难以继续优化了。

原文的后半部分提出了两种半动态模型(semi-dynamic model),并据此初步探讨了动态情形下SUPS问题的求解。

\begin{description}
    \item[滑窗模型] 每次支持的操作为,删除最左端的一位,或在最右端添加一位。滑窗模型构建了一个不断向右移动的窗,期间窗的大小\footnote{原文中窗的大小用$W$表示,本文也采取这种规范。}可能发生变化。
    \item[替换模型] 每次支持替换串中的一位为字母表\footnote{原文中字母表的大小用$\sigma$表示,本文也采取这种规范。}中的另一个字符。
\end{description}

原文中为滑窗模型提出了一个相当高效的算法,支持在每次操作后以均摊的$O(\log\sigma + \log\log W)$时间调整数据结构,并在$O(\log\log W)$求解任何SUPS问题。
原文中为替换模型提出的算法也相当高效,可以在$O(\log n \log\log n + k\log\log n)$时间\footnote{已包含调整数据结构的时间。}内解决任意$k$个SUPS问题。
这两种算法的空间复杂度和预处理过程的时间复杂度与静态模型一致。

在半动态模型下,被操作位的位置和操作方式都受到较大限制,因而很难满足实际应用的需求。本文基于原文中的两个半动态模型提出全动态模型,支持对原串进行如下三种操作:
\begin{itemize}
    \item 对原串中任一个间隔,以及开始处和终止处,插入一个新位
    \item 将原串中任一位删除
    \item 将原串中任一位替换,即用字母表中的其他字母来取代它
\end{itemize}
