
回文串是从前向后和从后向前读取结果相同的序列。由于其独特的性质以及在模式识别、数据压缩和DNA序列分析中的广泛应用,在生物信息学、计算机科学等领域,
对回文串的研究不断。本节探讨了回文串的主要研究方向,包括但不限于极小唯一回文子串、最短唯一回文子串和极大回文子串。

\subsection{极小唯一回文子串}\label{subsec:mups}

极小唯一回文子串(MUPS)的定义和性质详见\ref{subsec:intro}。求解MUPS的算法设计策略总结如下:
\begin{itemize}
    \item 基于后缀树。对输入的串构建后缀树,从后缀树中获取所有极小唯一的子串,再从中提取回文串\cite{Gusfield_1997}。
    这种策略需要消耗$O(n)$的时间以完成构建后缀树的预处理\cite{Ukkonen95}。
    \item 基于动态规划。使用一个动态规划的表格标记出所有回文子串,再依次检查每个回文子串是否唯一。时间复杂度为$O(n)$\cite{Manacher75}。
\end{itemize}

\subsection{极大回文子串}\label{subsec:mp}

极大回文子串(MP)的定义和性质详见\ref{subsec:intro}。求解MP的算法设计策略总结如下:
\begin{itemize}
    \item 使用Manacher算法\cite{Manacher75}。这是一个求解MP的在线(on-line)算法\footnote{即支持一位一位地按顺序输入串,运行过程中的结果是对于已输入串的问题解。},可以在线性时间内获得结果。
    \item 使用Eertree (Palindromic Tree)\cite{RubinchikS18}。这种数据结构可以储存所有互异回文子串,以便快速求解MP。
\end{itemize}

\subsection{最短唯一回文子串}\label{subsec:sups}

最短唯一回文子串(SUPS)的定义和性质详见\ref{subsec:intro}。求解SUPS问题,可以使用Inoue算法\cite{Inoue2018},详见\ref{subsec:inoue};
以及原文\cite{Mieno2024}对Inoue算法的改进算法,详见\ref{subsec:progress}和\ref{subsec:example}。

\subsection{回文分解}\label{subsec:factorize}
回文分解(palindromic factorization)是将一个串划分为最少数量的回文串的过程,常用于数据压缩和文本挖掘。借助Eertree数据结构,
k-分解问题可以在$O(kn)$时间内完成\cite{RubinchikS18}。
